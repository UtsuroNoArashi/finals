\documentclass{beamer}
\usepackage{../preamble/custom}

\setbeamertemplate{bibliography item}[text]

\title{Maximal repetition \& ``Runs'' conjecture}
\author{Riccardo Lo Iacono}
\begin{document}
    \begin{frame}
        \maketitle
    \end{frame}

    \begin{frame}{Definizioni preliminarie}
        \textbf{Definizione: } una terna \(r = (i, j, p)\) è una ripetizione 
        massimale (o \emph{run}) di una qualche stringa \(s\),
        se il più piccolo periodo \(p\) di \(s[i, j]\) è tale che 
        \(\abs{s[i, j]} = j - i \ge 2p\). 
        \vskip 10pt 
        \textbf{Esempio:} sia considerata la stringa \(s = aababaababb\).
        Si ha che gli intervalli \([1,2] = a^{2} , [6,7] = a^{2}, 
        [10,11] = b^{2}, [2, 6] = (ab)^{5\slash 2}, [7, 10] = (ab)^{2},
        [4, 9] = (aba)^{2}, [1, 10] = (aabab)^{2}\)
        hanno rispettivamente periodo, 1, 2, 3, 5.
        \vskip 10pt 
        Sia posto \(\rho(n)\) il numero di ripetizioni massimali.
        Inoltre, sia \(Runs(s)\) l'insieme delle run in \(s\). 
    \end{frame}

    \begin{frame}{Background storico}
        \emph{Kolpakov \emph{e} Kucherov}\footnotemark[1]
        dimostrano \(\rho(n) = \bigO{n}\).
        \vskip 10pt
        \textbf{Congettura: } \(\rho (n) < n\).

        \footnotetext[1]{R. M. KUCHEROV {\footnotesize AND} G. KOLPAKOV,
                 \emph{Finding maximal repetition in a word in linear time.}
         }
    \end{frame}
           
    \begin{frame}{Punti chiave della discussione}
        \begin{itemize}
            \item Dimostrazione della runs conjecture.
            \item Soluzione algoritmica per il calcolo delle ripetizioni 
                massimali in \(\bigO{n}\).
        \end{itemize}
    \end{frame}

    \begin{frame}{Notazione}
        \begin{itemize}
            \item \(\Sigma\) è un alfabeto finito di simboli
            \item \(s \in \Sigma^{*}\) è una stringa, la cui lunghezza è \(\abs{s}\)
            \item \(s[i]\) è l'iesimo carattere di \(s\), 
                \(s[i,j]\) è la sotto-stringa compresa tra gli indici \(i, j\)
                inclusivamente, \(i, j \in \range{1}{\abs{s}}\)
            \item \(p \in \mathbb{N} \text{ periodo di } s 
                \iff s[i] == s[i + p],  1 \le i \le \abs{s} - p\) 
            \item \(\mathcal{I}\) insieme di intevalli, 
                \(Beg(\mathcal{I})\) posizioni iniziali degli intervalli in \(\mathcal{I}\)
            \item \(\prec_{0}\) ordine totale su \(\Sigma\) e ordine lessicografico
                indotto su \(\Sigma^{*}\), \(\prec_{1}\) ordine rovesciato.
        \end{itemize}
    \end{frame}

    \begin{frame}
        \textbf{Esempio: } sia \(s = babbabbab\).
        Si osserva facilmente che le ripetizioni massimali in essa sono quelle 
        in \emph{Figura \ref{fig:1}}.
        \begin{figure}[!h]
            \centering
            \begin{tikzpicture}
                \foreach \char/\i in {b/0, a/1, b/2, b/3, a/4, b/5, b/6, a/7, b/8} {
                    \node at (\i, 0) {\char};
                }    

                \draw[ |-| ] (2, .5) -- (3, .5); 

                \draw[ |-| ] (5, .5) -- (6, .5);

                \draw[ |-| ] (0, -.5) -- (2.5, -.5);
                \draw[ -| ] (2.5, -.5) -- (5.5, -.5);
                \draw[ -| ] (5.5, -.5) -- (8, -.5);

                \node at (2.5, .5) [anchor =south] {\(r_{2}\)};
                \node at (5.5, .5) [anchor = south] {\(r_{3}\)};
                \node at (0, -.5) [anchor = east] {\(r_{1}\)};

            \end{tikzpicture}
            \caption{Esempio di ripetizioni massimali}
            \label{fig:1}
        \end{figure}
        Segue che 
            \[
                Runs(babbabbab) = \{(1, 9, 3), (3, 4, 1), (6, 7, 1)\} 
            \]
    \end{frame}

    \begin{frame}{Lyndon words \&  L-roots}
        \textbf{Definizione} (Lyndon word): una stringa non vuota \(s \in \Sigma^{*}\)
        è detta essere una \emph{Lyndon word}, rispetto a \(\prec\), 
        se \(s \prec u\), per ogni \(u\) suffisso proprio di \(s\).
        \vskip 5pt
         Definizione equivalente è la seguente.
        \vskip 5pt
        \textbf{Definizione: } data una stringa \(s\) di lunghezza \(n\) e 
        \(i \in [1, n]\), la stringa \(s_{i}s[1, i - 1]\) è detta 
        \emph{shift ciclico} di \(s\), non banale se \(i > 1\).
        Una Lyndon word è una stringa non vuota che è lessicograficamente minore
        di ogni suo shift ciclico non banale.
        \vskip 10pt 
        \textbf{Definizione} (L-root): data \(r = (i, j, p)\) una run per una 
        qualche stringa \(s \in \Sigma^{*}\),
        un intervallo \(\lambda = [i_{\lambda}, j_{\lambda}]\) è detto essere
        \emph{L-root} di \(r\) rispetto \(\prec\) se 
        \(i \le i_{\lambda} \le j_{\lambda} \le j\)
        e \(s[i_{\lambda}, j_{\lambda}]\) è una Lyndon word.
    \end{frame}

    \begin{frame}{``Runs'' Theorem}
        Sia \(\hat{s} = s\textdollar, \textdollar \notin \Sigma\).
        % Prima osservazione è che: se consideriamo la più lunga Lyndon word,
        % rispetto \(\prec_{0} \text{e} \prec{1}\) che inizia a un dato \(i\),
        % uno dei due avrà lunghezza 1.

        \textbf{Lemma 1:} per ogni stringa \(s\) e posizione \(i\),
        sia \(\ell \in \{0, 1\} \), tale che \(\hat{\omega}[k] \prec_{\ell} 
        \hat{s}[i], \text{per } k = \min \{ k' \, \vert \, \hat{s}[k'] \ne 
        \hat{s}[i], k' > i\}\). Allora \(l_{\ell}(i) = [i, i]\) e 
        \(l_{\overline{\ell}}(i) = [i, j], \text{per qualche } j > i\).
        \vskip 10pt 
        % Seconda osservazione è che: data una ripetizione \(r\),
        % esiste un ordine \(\prec_{\ell_{r}} \in \{\prec_{0}, \prec_{1}\}\)
        % tale che la L-root della runc rispetto \(\prec_{\ell_{r}}\)
        % coincide con la longest Lyndon word rispetto \(\prec_{\ell_{r}}\)
        % che inizia in quella posizione.
        \textbf{Lemma 2: } sia \(r = (i, j, p)\) una ripetizione massimale in una stringa
        \(s\), sia inoltre \(\ell_{r} \in \{0, 1\}\) tale che 
        \(\hat{s}[j + 1] \prec_{\ell_{r}} \hat{s}[j + 1 -p]\).
        Allora, ogni L-root \(\lambda = [i_{\lambda}, j_{\lambda}]\) di \(r\)
        rispetto \(\prec_{\ell_{r}}\) è uguale a \(l_{\ell_{r}}(i_{\lambda})\).
   \end{frame}

   \begin{frame}
       Sia \(B_{r}\) l'insieme delle L-root tali da soddisfare \emph{Lemma 2},
       allora vale quanto segue.

       \textbf{Lemma 3: } per ogni coppia di ripetizioni massimali \(r, r'\), 
       con \(r \ne r'\), \(Beg(B_{r}) \cap Beg(B_{r'}) = \varnothing\).

       Da ciò, poiché 
       \[
           \abs{Beg(B_{r})} = \abs{B_{r}} \ge \floor{e_{e} - 1} \ge 1
       \]
       e inoltre, dato che \(1 \notin Beg(B_{r})\) per ogni run \(r\), regge
       \(\sum_{r \in Runs(s)}{\abs{B_{r}}} = \sum_{r \in Runs(s)}{\abs{Beg(B_{r})}}
       \le \abs{s} - 1\).

       \textbf{Teorema: } \(\rho(s) < n\).
   \end{frame}
\end{document}
